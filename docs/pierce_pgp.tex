\documentclass[12pt]{article}
\usepackage[utf8x]{inputenc}
\usepackage{algorithm}
\usepackage{algpseudocode}
\usepackage{bm}
\usepackage{cite}
\usepackage{hyperref}
\usepackage{amsmath}

\title{Pierce's pattern generation procedure\thanks{Excerpt from Pierce~J.~F. 
Some large-scale production scheduling problems in the paper industry. 
--- Englewood Cliffs : Prentice-Hall, 1964. --- 255 p.}}
\date{}

\begin{document}
\maketitle

Suppose that a mill has recieved customers' orders for $r_1, r_2, \ldots, r_n$
rolls of a certain grade of paper, these rolls to be of widths
$w_1, w_2, \ldots, w_n$ and (the same) diameter $D$. The mill has one paper 
machine which can manufacture the desired grade, this machine producing rolls
of width $L$ (where $L \geq w_i$ for all $i$). Since customer widths demanded 
are smaller than, or equal to, the production width of the paper machine, the
production scheduler tries to find combinations of customers' 
widths\footnote{To identify such combinations Pierce extensively used the term 
``process''. Today it is more common to use term ``pattern'' instead.} with 
which to fill out the width $L$ of the paper machine rolls. Usually there will 
be a ``side roll'' of trim loss left over from such combinations which is 
repocessed or disposed of in some other manner. The paper trim problem, then, 
is to find trimming combinations of customer widths and to determine the number 
of machine rolls to be produced and cut according to each combination~--- so as 
to satisfy the customers' demand most efficiently \cite[p.~13]{bib:pierce64}.

[...] A \textit{dominating process} is defined to be one for which
\begin{equation}\label{eq:dominating_pattern}
 d_k=\left( L - \sum_{i}a_{ik}w_i\right)<\min_{i}(w_i) 
\end{equation}

where [...] $d_k$ is the trim loss for process $k$. Any process not 
satisfying \eqref{eq:dominating_pattern} is then dominated in the sense that
there exists at least one other process for the same paper machine which will
supply at least one more customer roll of some width (and the same number of 
all other widths), for each machine roll manufactured. [...] The matrix of all
possible processes for a problem will be termed the \textit{exhaustive process}
matrix, $\bm{A_E}$; the matrix containing  only the dominating processes will
be called the \textit{dominating} matrix $\bm{A_D}$ \cite[p.~18]{bib:pierce64}.

[...] generation procedure is as follows, where customer widths
 $w_1$, $w_2$, $\ldots$, $w_n$ are arranged in descending order:
\begin{algorithm}[ht]
    \algrenewcommand{\alglinenumber}[1]{\bfseries Step {#1}.}
    \renewcommand{\Statex}{\item[\hphantom{\bfseries Step \arabic{ALG@line}.}]}
    \begin{algorithmic}[1]
    \smallskip
    \State Set \(a_1=\min{\left( \left\lfloor \frac{L}{w_1} \right\rfloor, r_1 \right)},
    a_2=\min{\left( \left\lfloor\frac{L-a_1w_1}{w_2}\right\rfloor, r_2 \right)}, 
    \ldots,\) 
    \Statex $a_n=\min{\left( \left\lfloor\frac{L-\sum\nolimits_{i=1}^{n-1}{a_iw_i}}{w_n}\right\rfloor, r_n \right)}$ 
    \Statex where $\lfloor g \rfloor$ is the largest integer contained in $g$.

    \smallskip
    \State Record process in matrix $\bm{A_D}$.
    
    \smallskip
    \State If there is an $i$, $1\leq i \leq n-1$, such that $a_i>0$ then let $j$ be 
    \Statex the largest such $i$, and go to \textbf{Step 4}.
    \Statex If there is no such $i$, then terminate procedure: all dominating
    \Statex processes have been enumerated and recorded.
    
    \smallskip
    \State Set $a'_1=a_1, a'_2=a_2, \ldots, a'_j=a_j-1$;
    \Statex $a'_{j+1}=\min{\left( \left\lfloor\frac{L-\sum\nolimits_{i=1}^{j}{a_i^{\prime}w_i}}{w_{j+1}}\right\rfloor, r_{j+1} \right)}, \ldots,$ 
    \Statex $a'_{n}=\min{\left( \left\lfloor\frac{L-\sum\nolimits_{i=1}^{n-1}{a_i^{\prime}w_i}}{w_n}\right\rfloor, r_n \right)}$ 
    \Statex Go to \textbf{Step 2}.
    \end{algorithmic} 
\end{algorithm}

[...] a procedure for generating the exhaustive matrix $\bm{A_E}$ is obtained 
from this procedure merely by changing the range in step 3 from 
$1\leq i \leq n-1$ to $1\leq i \leq n$. The resulting procedure will enumerate 
all processes, regardless of the ordering of customer width 
\cite[p.~44]{bib:pierce64}.

\begin{thebibliography}{0}
\bibitem{bib:pierce64} Pierce~J.~F. Some large-scale production scheduling
problems in the paper industry. --- Englewood Cliffs : Prentice-Hall, 1964. 
--- 255 p.
\end{thebibliography}
\end{document}